\chapter{Why UUIDs as primary keys instead of auto-increment IDs?}

Designs of relational databases must have a secure out of conflicts way of identifying entities inside the database. One possible solution for this is using auto-increment functions of the database to provide this secure out of conflicts way of identifying entities.

Using the auto-increment functions, though, implies a huge problem: the information has to be inserted in a sequential way so that the foreign keys are built by the database manager (PostgreSQL in this case).

A solution implemented in the data model of the \acrshort{gsdm} is using \acrshort{uuid}s (version 1).

The version 1 of the UUID (according to RFC 4122) is composed by the following fields: 

\begin{lstlisting}[breaklines=true, style=c, caption={XML input example for showing the values structure management.}]

        time_low                the first 32 bits of the UUID
        time_mid                the next 16 bits of the UUID
        time_hi_version         the next 16 bits of the UUID
        clock_seq_hi_variant    the next 8 bits of the UUID
        clock_seq_low           the next 8 bits of the UUID
        node                    the last 48 bits of the UUID

        time                    the 60-bit timestamp
        clock_seq               the 14-bit sequence number
        
\end{lstlisting}

Where the implementation of Python of this version according to the RFC is thread safe due to the following code:

\begin{lstlisting}[breaklines=true, style=c, caption={Python code showing how the algorithm provides a mechanism for using UUIDs in a thread safe manner.}]

    global _last_timestamp
    import time
    nanoseconds = int(time.time() * 1e9)
    # 0x01b21dd213814000 is the number of 100-ns intervals between the
    # UUID epoch 1582-10-15 00:00:00 and the Unix epoch 1970-01-01 00:00:00.
    timestamp = int(nanoseconds/100) + 0x01b21dd213814000
    if _last_timestamp is not None and timestamp <= _last_timestamp:
        timestamp = _last_timestamp + 1
    _last_timestamp = timestamp
    
\end{lstlisting}

As \_last\_timestamp is global, this is visible to all the threads of the same process so that they will never see the same timestamp value.

But, this is still not multiprocessing safe. For making the UUIDs creation multiprocessing safe, the UUID has been enforced with the \acrshort{pid} of the process as the "node" field and a random number as the "clock\_seq".

So, this characteristic joint with the addition of the PID warranties the uniqueness of the UUIDs generated inside the GSDM.