\chapter{Create analysis}

In the following chapter, the process of generating analyses using the \acrshort{eboa} component is shown.

The explanation is based on an example for the analysis of the planning of the mission Sentinel-2.

The example will use the python interface of the component, as the target version of the \acrshort{eboa} used in this explanation is the 0.1.0 and it is the only interface available.

\section{Structure of the folders and location of the analyses inside the project}

The analysis generators are being located for the moment in the folder:

\begin{lstlisting}[breaklines=true, style=bash]]

src/analysis

\end{lstlisting}

Inside this folder there is the folder s2 where the example explained here can be located:

\begin{lstlisting}[breaklines=true, style=bash]]

src/analysis/s2

\end{lstlisting}

There, the code of the processor, a folder for automated tests and a folder for input examples can be found:

\begin{lstlisting}[breaklines=true, style=bash]]

mission_planning.py

\end{lstlisting}

It is recommended to read the code inside the file mission\_planning.py.

\section{Analysis code}

The analyses are generated in xls files using the library \href{https://openpyxl.readthedocs.io/en/stable/}{openpyxl}.

For creating analyses the \acrshort{eboa} component offers:

\begin{itemize}
\item The query submodule for accessing the data inserted into the database which \acrshort{api} can be seen in the section \ref{\detokenize{eboa.engine:eboa-engine-query-module}}.
\item The analysis subpackage with some helpers for generating the desired analyses which can be seen in the section \ref{\detokenize{eboa.analysis::doc}}.
\end{itemize}

The analysis code has a main method:

\begin{lstlisting}[breaklines=true, style=python]]
def generate_analysis(file_path, begin, end):
    """
    Method to generate the specific analysis into the file received as
    argument and following the period between the begin and end dates
    
    :param file_path: path to the file where the analysis is going to be stored
    :type file_path: str
    :param begin: start date of the period
    :type being: date
    :param end: stop date of the period
    :type end: date
    """
\end{lstlisting}

Which should be provided as interface of the analysis as in future versions it will be the main entry point to the analysis.

This method shall generate an xls file containing the interesting information falling into the specified period obtained from the DDBB using the query \acrshort{api} of the \acrshort{eboa}

For the version of this example of an analysis generator, this method is called from the main entry point of the same phyton file:

\begin{lstlisting}[breaklines=true, style=python]]
def generate_analysis(file_path, begin, end):
    ...

if __name__ == "__main__":
    ...

    returned_value = command_process_file(file_path)
\end{lstlisting}

The following list includes a brief description of main operations available for generating the analysis into an xls file:

\begin{itemize}
\item Importing the main modules:
\begin{lstlisting}[breaklines=true, style=python]]
import openpyxl
from openpyxl import Workbook
from openpyxl.styles import Font, PatternFill, Border, Side
from eboa.engine.query import Query
from eboa.analysis.functions import adjust_column_width
\end{lstlisting}
\item Obtain the interesting information from the database:
\begin{lstlisting}[breaklines=true, style=python]]
    imaging_events_and_linked = query.get_linked_events_join(gauge_name_like = {"str": "CUT_IMAGING%", "op": "like"}, start_filters = [{"date": end, "op": "<"}], stop_filters = [{"date": begin, "op": ">"}], link_names = {"list": ["RECORD_OPERATION", "COMPLETE_IMAGING_OPERATION"], "op": "in"})
\end{lstlisting}
\item Creating the workbook:
\begin{lstlisting}[breaklines=true, style=python]]
    # Imaging plan
    ws = workbook.create_sheet("Imaging plan")

    # Insert headings into the worksheet
    ws.append(["Satellite", "Gauge", "Start", "Stop", "Duration (m)", "Event uuid", "Ingestion time", "Parameters", "NPPF file", "DIM version"])
\end{lstlisting}
\item Adjust column's size:
\begin{lstlisting}[breaklines=true, style=python]]
    adjust_column_width(ws)
\end{lstlisting}
\end{itemize}

\section{Manual execution of the analysis}

The analysis can be manually executed as follows:

\begin{lstlisting}[breaklines=true, style=bash]]

$ cd /vagrant/src/analysis/s2
$ python3 mission_planning.py -f /tmp/test.xls -b "20180803T110000" -e "20180820T140000"

\end{lstlisting}
