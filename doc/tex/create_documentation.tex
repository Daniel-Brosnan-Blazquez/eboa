\chapter{Generate this documentation}

In the following chapter, the process of generating this documentation is explained

The documentation has been created using latex for automation on adding content from the source code and auto-generated documentation.

For having all the needed packages, the documentation has to be generated in a ubuntu virtual machine created by vagrant.

Follow the next steps for generating the documentaion:

\begin{enumerate}

\item Create the ubuntu virtual machine

\begin{lstlisting}[breaklines=true, style=bash]

$ vagrant up ubuntu

\end{lstlisting}

\item Enter into the machine

\begin{lstlisting}[breaklines=true, style=bash]

$ vagrant ssh ubuntu

\end{lstlisting}

\item Go to the proper folder and generate the documentation

\begin{lstlisting}[breaklines=true, style=bash]

$ cd /vagrant/doc/tex
$ ./generate_doc.sh -f [EBOA]_[0.1.0].pdf -k

\end{lstlisting}

\end{enumerate}

Note: if there is any error shown by the command generate\_doc.sh, check the log inside build/doc.log.

Follow the next steps for accessing the documentation for inspection:

\begin{enumerate}

\item Check the ssh configuration for the ubuntu virtual machine:

\begin{lstlisting}[breaklines=true, style=bash]

$ vagrant ssh-config ubuntu

\end{lstlisting}

\item Copy the documentation to your local environment (tested on Ubuntu, use your file transfer for better convinience):

\begin{lstlisting}[breaklines=true, style=bash]

$ vagrant ssh-config ubuntu
Host ubuntu
  HostName 127.0.0.1
  User vagrant
  Port 2200
  UserKnownHostsFile /dev/null
  StrictHostKeyChecking no
  PasswordAuthentication no
  IdentityFile PATH_TO_PRIVATE_KEY/private_key
  IdentitiesOnly yes
  LogLevel FATAL
  ForwardX11 yes

$ scp -i PATH_TO_PRIVATE_KEY/private_key -P 2200 vagrant@127.0.0.1:/vagrant/doc/tex/*pdf /tmp/

\end{lstlisting}

\end{enumerate}
