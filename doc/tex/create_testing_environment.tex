\chapter{EBOA testing environment}\label{c:install_eboa}

Inside the repository of the \acrshort{eboa} component there is a \href{https://www.vagrantup.com/}{Vagrant} configuration file so that the environment for testing can be created automatically.

For doing so, follow the next steps:

\begin{enumerate}

\item Clone the repository

\begin{lstlisting}[breaklines=true, style=bash]

$ git clone https://bitbucket.org/dbrosnan/eboa.git

\end{lstlisting}

\item Enter into the repository folder

\begin{lstlisting}[breaklines=true, style=bash]

$ cd (PATH_TO_EBOA)/eboa

\end{lstlisting}

\item Run vagrant (see installation instructions in the referenced link)

\begin{lstlisting}[breaklines=true, style=bash]

$ vagrant up centos

\end{lstlisting}

\end{enumerate}

This will create a virtual machine with all you need to test the EBOA component.

In order to check that the component has been correctly installed, run the following commands to execute the automated tests:

\begin{lstlisting}[breaklines=true, style=bash]]

$ vagrant ssh centos
$ cd /vagrant/src/
$ py.test -v --cov-report html:tests/tmp/code_coverage_analysis --cov-branch --cov=eboa --cov=rboa tests

\end{lstlisting}

In order to check the coverage report, check section \ref{x11_forwarding} and install firefox and open the code coverage analysis report:

\begin{lstlisting}[breaklines=true, style=bash]]

$ sudo yum install firefox
$ firefox tests/tmp/code_coverage_analysis/index.html

\end{lstlisting}

\section {Connecting to vagrant with X11 forwarding}\label{x11_forwarding}

Read these links for detailed information:

\begin{itemize}

\item \href{https://coderwall.com/p/ozhfva/run-graphical-programs-within-vagrantboxes}{Run graphical programs within Vagrantboxes} 
\item \href{https://www.cyberciti.biz/faq/how-to-fix-x11-forwarding-request-failed-on-channel-0/}{Install X authority file utility} 

\end{itemize}

Allow X-Forwarding in your Vagrantfile:

To use X-Forwarding, you first need to allow it from within your Vagrantfile, like this:

\begin{lstlisting}[breaklines=true, style=bash]]

Vagrant.configure(2) do |config|
  ...
  config.ssh.forward_x11 = true
end

\end{lstlisting}

To be able to execute GUIs in the ssh session, you have to install xauth inside the virtual machine created:

\begin{lstlisting}[breaklines=true, style=bash]]

sudo yum install xauth

\end{lstlisting}

Now on your host run this:

\begin{lstlisting}[breaklines=true, style=bash]]

$ vagrant ssh-config
Host some_site
  HostName 127.0.0.1
  User vagrant
  Port 2222
  UserKnownHostsFile /dev/null
  StrictHostKeyChecking no
  PasswordAuthentication no
  IdentityFile vagrant.d/insecure_private_key
  IdentitiesOnly yes
  LogLevel FATAL
  ForwardX11 yes

\end{lstlisting}

Now you can ssh with X11 forwarding:

\begin{lstlisting}[breaklines=true, style=bash]]

ssh -X -p 2222 vagrant@localhost -i vagrant.d/insecure_private_key

\end{lstlisting}

