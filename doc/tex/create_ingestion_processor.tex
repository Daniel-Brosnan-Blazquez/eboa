\chapter{Create ingestion processor}

In the following chapter, the process of creating a processor for ingesting data to the \acrshort{eboa} module is explained.

The explanation is based on an example for ingesting Sentinel-2 data from the planning system (data received on NPPF files, xml formatted).

The example will use the python interface of the component, as the target version of the \acrshort{eboa} used in this explanation is the 0.1.0 and it is the only interface available.

\section{Installing EBOA on vagrant}

See the details in section \ref{c:install_eboa}.

\section{Structure of the folders and location of ingestion processors inside the project}

The ingestion processors are being located for the moment in the folder:

\begin{lstlisting}[breaklines=true, style=bash]]

src/ingestions

\end{lstlisting}

Inside this folder there is the folder s2 where the example explained here can be located:

\begin{lstlisting}[breaklines=true, style=bash]]

src/ingestions/s2/ingestion_nppf

\end{lstlisting}

There, the code of the processor, a folder for automated tests and a folder for input examples can be found:

\begin{lstlisting}[breaklines=true, style=bash]]

ingestion_nppf.py
input_files
tests

\end{lstlisting}

It is recommended to read the code inside the file ingestion\_nppf.py.

\section{Processor code}

For creating ingestion processors the \acrshort{eboa} component offers some helpers through the submodule ingestion which can be seen in the section \ref{\detokenize{eboa.engine:eboa-engine-ingestion-module}}.

The processor code has a main method:

\begin{lstlisting}[breaklines=true, style=python]]
def process_file(file_path):
    """Function to process the file and insert its relevant information
    into the DDBB of the eboa

    :param file_path: path to the file to be processed
    :type file_path: str
    """
\end{lstlisting}

Which should be provided as interface of the processor as in future versions it will be the main entry point to the processor.

This method shall extract from the file passed by parameter all the interesting information to the system.

The data extracted has to be formatted with the structure of a python dictionary which can be validated against a schema implemented in the engine side of the \acrshort{eboa} (method eboa.engine.engine.validate\_data). This data will be validated before treatement by the engine.

An example of the data structure, that has to follow the data to be inserted into the \acrshort{eboa}, can be seen in appendix \ref{ap:example_data_structure}.

For the version of this example of an ingestion processor, this method is called from the same phyton file from another method that is called from the main entry point to the file:

\begin{lstlisting}[breaklines=true, style=python]]
def command_process_file(file_path):
    # Process file
    data = process_file(file_path)
    ...

if __name__ == "__main__":
    ...

    returned_value = command_process_file(file_path)
\end{lstlisting}

Once the data has been extracted the method command\_process\_file(file\_path) will interface with the Engine of the \acrshort{eboa} to insert the data:

\begin{lstlisting}[breaklines=true, style=python]]
    # Process file
    data = process_file(file_path)

    engine = Engine()
    # Validate data
    filename = os.path.basename(file_path)

    # Treat data
    returned_value = insert_data_into_DDBB(data, filename, engine)
\end{lstlisting}

\section{Extraction of data}

\subsection{Extraction from an XML file}

The example provided performs the extraction of data from an XML file. To do so, the module uses the library \href{https://pypi.org/project/lxml/}{lxml}.

The following list includes a brief description of main operations available using this library:

\begin{itemize}
\item Importing the module:
\begin{lstlisting}[breaklines=true, style=python]]
# Import xml parser
from lxml import etree
\end{lstlisting}
\item Obtain Xpath object for reading the XML using XML paths:
\begin{lstlisting}[breaklines=true, style=python]]
    file_name = os.path.basename(file_path)
    parsed_xml = etree.parse(file_path)
    xpath_xml = etree.XPathEvaluator(parsed_xml)
\end{lstlisting}
\item Get nodes of an XML:
\begin{lstlisting}[breaklines=true, style=python]]
    record_operations = xpath_xml("/Earth_Explorer_File/Data_Block/List_of_EVRQs/EVRQ[RQ/RQ_Name='MPMMRNOM' or RQ/RQ_Name='MPMMRNRT']")
\end{lstlisting}
\item Perform operations with one node:
\begin{lstlisting}[breaklines=true, style=python]]
    generation_time = xpath_xml("/Earth_Explorer_File/Earth_Explorer_Header/Fixed_Header/Source/Creation_Date")[0].text.split("=")[1]
\end{lstlisting}
\end{itemize}

\section{Manual execution of the processor}

The processor can be manually executed passing the file path to be processed as follows:

\begin{lstlisting}[breaklines=true, style=bash]]

$ cd /vagrant/src/ingestions/s2/ingestion_nppf
$ python3 ingestion_nppf.py -f input_files/S2B_OPER_MPL__NPPF__20180727T110000_20180813T140000_0001.EOF

\end{lstlisting}

\section{Automated tests}

Inside the folder tests, there should be a python file which would cover the unit testing of the processor. For the example explained in this page, to execute the tests, run the following command:

\begin{lstlisting}[breaklines=true, style=bash]]

$ cd /vagrant/src/ingestions/s2/ingestion_nppf
$ py.test -v --cov-report html:tests/tmp/code_coverage_analysis --cov=ingestions tests/

\end{lstlisting}

Then, the code coverage analysis may be checked (following the instructions shown in the section \ref{x11_forwarding} and installing firefox) with the following command:

\begin{lstlisting}[breaklines=true, style=bash]]

$ firefox tests/tmp/code_coverage_analysis/index.html

\end{lstlisting}

\subsection{Creating the tests}

For validating the correct behaviour of the ingestion processor, automated tests should be provided which would cover completely the code of the ingestion processor.

For doing this, the provided solution makes use of the library unittest. With this library the utility py.test can detect all the tests inside a project, execute them and provide analysis on the failures.

The following list includes a brief description of main operations available using this library:

\begin{itemize}
\item Import the module:
\begin{lstlisting}[breaklines=true, style=python]]
import unittest
\end{lstlisting}
\item Define class for testing:
\begin{lstlisting}[breaklines=true, style=python]]
class TestEngine(unittest.TestCase):
\end{lstlisting}
\item Define the main method which will be executed every time before a unit test is executed:
\begin{lstlisting}[breaklines=true, style=python]]
    def setUp(self):
        # Create the engine to manage the data
        self.engine_eboa = Engine()
        self.query_eboa = Query()

        # Create session to connectx to the database
        self.session = Session()

        # Clear all tables before executing the test
        for table in reversed(Base.metadata.sorted_tables):
            engine.execute(table.delete())
\end{lstlisting}
\item Define unit tests:
\begin{lstlisting}[breaklines=true, style=python]]
    def test_insert_insert_nppf(self):
        filename = "NPPF_CONTAINING_ALL_DATA_TO_BE_PROCESS.EOF"
        file_path = os.path.dirname(os.path.abspath(__file__)) + "/inputs/" + filename

        ingestion.command_process_file(file_path)

        # Check that events before the queue deletion are not inserted
        events_before_validity_period = self.session.query(Event).filter(Event.stop < "2018-07-20T13:40:00.000").all()

        assert len(events_before_validity_period) == 0
\end{lstlisting}
\end{itemize}
